\documentclass[fs, footer]{latex4ei}
\usepackage[european]{circuitikz}
\usepackage{tikz}
\begin{document}
\begin{multicols*}{4}
\fstitle{Grundlagen der \\ Energie- \\ übertragungstechnik}

 \ctikzset{bipoles/length=0.7cm}
\section{Rechnerische Behandlung des Drehstromsystems}

\sectionbox{
	 \subsection{Spannung $u$}
	
	 Augenblickswert:
	 $u(t) = \hat u \cos ( \omega t + \varphi_u) = U \sqrt 2 \cos (\omega t + \varphi_u)$ \\ 

	 Scheitelwert / Amplitude:  $\hat u$ \\ 
 
	 Effektivwert (allg.): $U_{\ir eff} = U = \sqrt{\frac{1}{T} \int \limits_{t_0}^{t_0 + T} u^2(\tau) \diff \tau}$ \\
	 $\ra $ bei sinusförmigen Größen: $U = \frac{\hat u}{\sqrt 2}$ \\ 

	 Phasenwinkel: $\varphi (t) = \omega t + \varphi_u$ mit $\omega = 2 \pi f = \frac{2 \pi}{T}$ \\ 

	 Nullphasenwinkel: $\varphi_u$ \\ 
}
\sectionbox{

	\subsection{Stromstärke $i$ }
	 Augenblickswert:
	 $i(t) = \hat i \cos ( \omega t + \varphi_i) = I \sqrt 2 \cos (\omega t + \varphi_i)$ \\ 

	 Scheitelwert / Amplitude:  $\hat i$ \\ 
 
	 Effektivwert (bei sinusförmigen Größen): $I = \frac{\hat i}{\sqrt 2}$ \\ 

	 Phasenwinkel: $\varphi (t) = \omega t + \varphi_i$ mit $\omega = 2 \pi f = \frac{2 \pi}{T}$ \\ 

	 Nullphasenwinkel: $\varphi_i$ \\ 

	 \textbf{Phasenverschiebungswinkel:} \\ 
	 $\varphi = \varphi_{ui} = \varphi_u - \varphi_i$ \\
	 
	  $0 < \varphi \le \pi$: Strom eilt der Spannung nach \\
	 $-\pi \le \varphi < 0$: Strom eilt der Spannung vor
}
\sectionbox{
	 \subsection{Symmetrische Komponenten} 

	 zerlegen eines Dreileiter-Drehstromnetz in unabhängige Systeme (Mit-, Gegen- und Nullsystem) \\


	 Entsymmetrierungsmatrix $\ma T$:

	 $\vect{\vec I_1 \\ \vec I_2 \\ \vec I_3} = \mat{1 & 1 & 1 \\ \vec a^2 & \vec a & 1 \\ \vec a & \vec a^2 & 1} \cdot \vect{\vec I_{(1)1} \\ \vec I_{(2)1} \\ \vec I_{(0)1}}$ \\

	 Symmetrierungsmatrix $\ma S$:

 	$\vect{\vec I_{(1)1} \\ \vec I_{(2)1} \\ \vec I_{(0)1}} = \frac{1}{3} \cdot \mat{1 & \vec a & \vec a^2 \\ 1 & \vec a^2 & \vec a \\ 1 & 1 & 1 } \cdot  \vect{\vec I_1 \\ \vec I_2 \\ \vec I_3} $ \\ \\ \\

 	Mitsystem:

 	\begin{circuitikz}[scale=0.8]
 	\draw (0,0) -- (1,0) to[R, o-o, l^=$\vec{Z_{(1)}}$, i^>=$\vec{I_{(1)}}$ ] (4,0) to[open, v^>=$\vec{U_{(1)}}$, o-o] (4, -1.5) -- (1, -1.5) to[short, o-] (0,-1.5) to[sV, v^<=$\vec{E_{1}}$] (0,0);
 	\end{circuitikz} \\

 	Gegensystem:

 	\begin{circuitikz}[scale=0.8]
 	\draw (0,0) -- (1,0) to[R, o-o, l^=$\vec{Z_{(2)}}$, i^>=$\vec{I_{(2)}}$ ] (4,0) to[open, v^>=$\vec{U_{(2)}}$, o-o] (4, -1.5) -- (1, -1.5) to[short, o-] (0,-1.5) to[short] (0,0);
 	\end{circuitikz} \\

 	Nullsystem:

 	\begin{circuitikz}[scale=0.8]
 	\draw (0,0) -- (1,0) to[R, o-o, l^=$\vec{Z_{(0)}}$, i^>=$\vec{I_{(0)}}$ ] (4,0) to[open, v^>=$\vec{U_{(0)}}$, o-o] (4, -1.5) -- (1, -1.5) to[short, o-] (0,-1.5) to[short] (0,0);
 	\end{circuitikz}

}
 	\section{Freileitungen und Kabel}



\subsection{Freileitungen}

\sectionbox{
\subsubsection{Durchhang der Freileitungsseile}

\symbolbox{
\begin{tabular}{cc}
 $f(x)$ & Durchhang an der Stelle $x$ \\
$G'$ & Gewichtskraft des Leiterseils pro Längeneinheit \\
$F_H$ & Horizontalzugkraft \\
$h$ & mittlere Höhe der Leiterseile
\end{tabular}
}


Durchhang: $f(x) = f_{\ir max} - \frac{G'}{2 F_h} x^2)$ mit $f_max = \frac{G'}{8 F_H} a^2$


mittlere Höhe: $h = h_{\ir Mast}  - 0,7 \cdot f_{\ir max}$ \\ 

Mindestabstand zum Erdboden (VDE 0210-1):


\symbolbox{
\begin{tabular}{cc}
$D_{\ir el}$ & elektrischer Grundabstand
\end{tabular}
} \\ 

Mindestabstand: $h_{\ir min} = 5 \si{\meter} + D_{\ir el}$ \\ 
}
\sectionbox{
\subsubsection{Bündelleiter}

\symbolbox{
	\begin{tabular}{cc}
	$r_B$ & Ersatzradius \\
	$n$ & Anzahl der Teilleiter \\
	$r$ & Seilradius eines Teilleiters \\
	$r_T$ & Teilkreisradius
	\end{tabular}
}
für einen Bündelleiter mit 4 Teilleitern: $r_T = \sqrt 2 \frac{a}{2}$

Ersatzradius $r_b$: $r_n = \sqrt[n]{\cdot r \cdot r^{n-1}_{T}} $

\textbf{Abstand zwischen den Außenleitern} \\ 
Ersatzabstand $D_{\ir ers}$ für Einfachsysteme: $D = \sqrt[3]{D_{12} \cdot D_{23} \cdot D_{31}}$ \\ \\
Ersatzabstand für Doppelsysteme mit $\gamma$-Verdrillung: \\
$D =  \sqrt[3]{D_{12} \cdot D_{23} \cdot D_{31} \cdot \frac{D_{12'} \cdot D_{23'} \cdot D_{31'}}{D_{11'} \cdot D_{22'} \cdot D_{33'}}}$
}
\sectionbox{
\subsubsection{Widerstandsbelag}

\symbolbox{
	\begin{tabular}{cc}
	$Q$ & Querschnitt \\
	$\alpha$ & Temperaturkoeffizient \\
	$\vartheta$ & Celsiustemperatur
	\end{tabular}
}

Widerstand pro Längeneinheit: $R_{20}' =  \frac{1}{\kappa_{20} Q}$ \\
mit Temperaturabhängigkeit: $R_{\vartheta}'= \frac{1}{K_{20} Q} \left[ 1+ \alpha ( \vartheta - 20 K) \right]$\\ 

im Bündelleiter: $R_B = \frac{1}{n} \cdot R$ mit $R$ ist Widerstand des Teilleiters
}
\sectionbox{
\subsubsection{Induktivitätsbelag}
Betriebsinduktivitätsbelag: $L_b' = (2 \ln \frac{D_{\ir ers}}{r} + \frac 1 2 ) \cdot 10^{-4} \si{\henry \per \kilo \meter}$ \\
für Bündelleiter: $L_b' = (2 \ln \frac{D_{\ir ers}}{r_B} + \frac{1}{2 n}) \cdot 10^{-4} \si{\henry \per \kilo \meter}$ 
}
\sectionbox{
\subsubsection{Kapazitätsbelag}
 $C_b ' = \frac{2 \pi \epsilon_0}{\ln \left( \frac{D_{\ir ers}}{r_B \sqrt{1 + \left( \frac{D_{\ir ers}}{2 h}\right)^2 } }\right)} $ \\ 


$D \ll 2h$ $C_b' = \frac{2 \pi \epsilon_0}{\ln \left( \frac{D_{\ir ers}}{ r_B} \right) }$ \\ 

}
\sectionbox{
\subsubsection{Ohmscher Querleitwert $G'$}

spezifische Arbeitsverluste: $P_V' = 3 \cdot \left( \frac{U_n}{\sqrt{3}} \right)^2 \cdot G_b' = U_n^2 \cdot G_b'$ \\

Betriebsableitbelag: $G_b' = \frac{P_V'}{U_n^2}$

}
\sectionbox{
\subsection{Kabel}


Querschnitt: $Q = (r_a^2 - r_i^2) \pi$ \\ 




\subsubsection{Widerstandsbelag $R'$}

\subsubsection{Induktivitätsbelag $L'$}
Betriebsinduktivitäsbelag: $L_b' = \frac{L_B}{l} = \left( 2 \ln \frac{D}{r}+ \frac{1}{2} \right) \cdot 10^{-4} \si{\henry \per \kilo \meter}$ \\

Betriebsreaktanzbelag: $X_b ' = \omega	L_b ' $ \\ 

Induktivitätsbelag eines Hohlleiters:

$\omega L_{b_{\ir HL}}' = ( 0,96 + 0,051 \frac{r_a - r_i}{r_a})$ für $0 < \frac{r_a - r_i}{r_a} < 0,6$
\\

\subsubsection{Kapazitätsbelag $C'$}

Radialfeldkabel: $C_b' = \frac{2 \pi \epsilon}{\ln \left( \frac{R_a}{R_i} \right )}$ \\ 


\subsubsection{Ohmscher Querleitwert}
$G_b' = \tan \delta \cdot \omega C_b'$
}
\section{Leitung im stationären und nichtstationären Betrieb}
\sectionbox{
\subsection{verlustlose Fernleitung}
 $\beta = \sqrt{ L' C' }$ \\ 
$\vartheta_{\ir nat} = \beta l $

Wellenwiderstand für verlustlose Leitungen: $Z_W = \sqrt{\frac{\omega L'}{\omega C'}}$ \\ 

Längsimpedanz (elektrisch lang): $\vec Z_l = Z_w \j \sin ( \beta l)$ \\

Queradmittanz (elektrisch lang): $\frac{Y_q}{2} = \frac{1}{Z_w} \j \tan \left( \frac{\beta l}{2} \right)$

natürliche Leistung: $P_{\ir nat} = \frac{U_n^2}{Z_w}$


Zweitorgleichung: $\mat{ \vec U_1  \\ \vec I_1 } = \mat{ \cos(\beta l) & Z_w \j \sin ( \beta l) \\ \frac{1}{Z_w} \j \sin (\beta l) & \cos ( \beta l)} \mat{\vec U_2  \\ \vec I_2 }$
} 
\sectionbox{
\subsection{Querkompensation}

\boxed{
Q_2 = \frac{P_{\ir nat}}{\sin ( \beta l )} \left[ \sqrt{1 - \left( \frac{p_2}{P_{\ir nat}} \sin (\beta l) \right)^2} - \cos (\beta l) \right]}
}
\sectionbox{
\subsection{Längskompensation}
$\frac{P_{\ir natK}}{P_{\ir nat}} = \sqrt{\frac{1 - k_q}{1-k_l}} $

Faustregel für die optimale Anzahl der Kondensatorbatterien: \\
$0 < k_l \le 0,5 \Ra n = 1 \Ra X_k = k_l \cdot 2 Z_w \sin (\beta \frac{l}{2})$ \\
$0,5 < k_l \le 0,67 \Ra n = 2 \Ra X_k = k_l \cdot \frac{3}{2} Z_w \sin (\beta \frac{l}{3})$ \\
$0,67 < k_l \le 0,75 \Ra n = 3 \Ra X_k = k_l \cdot \frac{4}{3} Z_w \sin (\beta \frac{l}{4})$ \\ \\ 

Komensationsblindleistung: $Q_K = 3 \cdot X_K \cdot I_K^2$

Leistungswinkel der kompensierten Leitung: $\vartheta_k = \beta_k l$

Grenzwinkel für Stabilität der Leitung: $\vartheta_{\ir Grenz} = 42^\circ$
$\vartheta = (\vartheta_M)_{\ir grenz} - (\vartheta_M + \vartheta_T)$ mit Transformatorwinkel $\vartheta_T \approx 3,5^\circ$, $\vartheta_M \approx 5,5^\circ$
}
\sectionbox{
\subsection{Wanderwellen}
}

\section{Transformatoren}

\sectionbox{
\subsection{Zweiwicklungstransformator}

 \ctikzset{bipoles/length=0.7cm}
\begin{circuitikz}[ scale =0.5]
\draw
(0,0) to[R, o-, i>^=$\vec I_1$, l=$R_1$] (3,0) to[L, -*, l=$\j X_{\sigma 1}$] (5,0 ) to[short, -*] (6,0) to[L, l=$\j X_{\sigma 2}^{\times}$] (8,0)
 to[R,l=$R_2^\times$, i^>=$\vec I_2^\times$, -o] (12,0) to[open, v^>=$\vec U_2^\times$] (12, -4) -- (6, -4) to[short, *-*] (5, -4) to[short, -o] (0, -4) to[open, v^<=$\vec U_1$] (0,0)
(5,0) to[R, l_=$R_{\ir Fe}$, i>_=$\vec I_{\ir h}$] (5, -4)
(6,0) to[L, l=$\j X_{\ir h}$, i>=$\vec I_{\ir Fe}$] (6, -4)
 ;
\end{circuitikz}

Übersetzung: $\textrm{ü} = \frac{w_1}{w_2}$ \\

$\vec I_2^\times = \frac{\vec I_2}{\textrm{ü}}$ mit $w_1, w_2$ sind Windungszahlen \\
$\vec U_2^\times = \textrm{ü} \vec U_2$  \\
$\vec Z^\times_2 = \textrm{ü}^2 \vec Z_2$ \\
$\vec U_2^\times \cdot \vec I_2^\times = \textrm{ü}  \vec U_2 \cdot \frac{\vec I_2}{\textrm{ü}} = \vec U_2 \cdot \vec I_2$


Bemessungsstrom: $\vec I_r = \frac{S_{rT}}{\sqrt{3} U_{rT}}$

Kupferverluste: $P_{\ir cu} = 3 U_{R_k} I_1 = 3 R_k J_1^2$
}


\subsubsection{Leerlauf}

Leerlaufstrom: $\vec I_0 = \vec I_{\ir Fe} + \vec I_h$

Hauptreaktanz: $X_h = \Im{\frac{\vec U_{r1}}{\vec I_h}}$ 

Eisenverluste (im Einphasentransformator): $R_{\ir Fe} = \frac{U_{r1}^2}{P_0} = \textrm{ü}^2 \cdot \frac{U^2_{r2}}{P_0}$

\subsubsection{Kurzschluss}

\begin{circuitikz}
\draw
(0,0) to[R, o-, i>^=$\vec I_1$, l=$R_k$] (1.5,0) to[L, l=$\j X_k$, i^>=$\vec I_2^\times$] (3,0) -- (3,-1) to[short, -o] (0, -1) to[open, v^<=$\vec U_1$] (0,0)
;
\end{circuitikz} \\ 

Kurzsschlussimpedanz: \\
$\vec Z_k = \frac{\vec U_k}{\vec I_{r1}} = (R_1 + R_2^\times) + \j (X_{\sigma1} + \X_{\sigma2}^\times) = R_k + \j X_k$ \\ \\ 

relative Kurzschlussspannung: \\
$u_k = \frac{U_k}{U_{r1}} = \frac{I_{r1} \cdot X_k}{U_{r1}} \cdot \left( \frac{U_{r1}}{U_{r1}} \right) = \frac{S_r \cdot X_k}{U_{r1}^2}$ 


Bezogener Spannungsfall: $u_x = \frac{X_k S_{rT}}{U_{r1}^2}$
\subsubsection{Bemessungsstrom}

\begin{circuitikz}
\draw
(0,0) to[R, o-, l=$R_k$, i>^=$\vec I_1$](1.5,0) to[L, -o, l=$\j X_k$, i^>=$\vec I_2^\times$] (3,0) to[open, v^>=$\vec U_2^\times$] (3,-1)
to[short, o-o] (0,-1)to[open, v^<=$\vec U_1$] (0,0)
 ;
\end{circuitikz}

\subsubsection{Übersetzung \textrm{ü}}

Übersetzung: $\textrm{ü} = \frac{w_1}{w_2}$ mit $w_1 :=$ Primärwicklung, $w_2 :=$ Sekundärwicklung \\

Leerlaufübersetzung $\textrm{ü}_0 = \frac{\vec U_1}{\vec U_2} \approx \frac{w_1}{w_2}$


\subsection{Drehstromtransformator}

Kurzschlussspannung: $u_k = \frac{U_{kT}}{U_{rT}} = \frac{\frac{U_{kT}}{\sqrt 3}}{\frac{U_{rT}}{\sqrt 3}} = \frac{X_k \cdot I_{rT}}{\frac{U_{rT}}{\sqrt 3}}$

Bemessungsleistung: $S_{rT} = \sqrt{3} \cdot U_{rT} \cdot I_{rT}$

Kurzschlussreaktanz: $X_k = \frac{u_k \cdot U_{rT}^2}{S_{rT}}$ 

$X_{k(Y)} = \frac{X_{k(\Delta)}}{3}$

\subsection{Wicklungsverschaltung}

\subsubsection{Sternschaltung}
\begin{circuitikz} [scale=0.5]
 \draw
(0,0) node[anchor=south] {U} to[L, v_>=$\vec U_{\ir U}$, o-] (0, -3) -- (1,-3) to[L, -o] (1,0) node[anchor=south] {V}
(1,-3) -- (2, -3) to[L, -o] (2, 0)  node[anchor=south] {W}
(3,0) to[short, -o] (3,0) node[anchor=south] {N};
\draw[dashed] (2, -3) -- (3, -3) to[short] (3,-0.1) 
 ;
 \end{circuitikz} 

\begin{circuitikz}
\draw[->] (0.8, -1) -- (0, 0) node[anchor=west] {$\vec U_{\ir W}$};
\draw[->](0.8, -1) -- (0, -2) node[anchor=west] {$\vec U_{\ir V}$};
\draw[->] (0.8, -1) -- (1.8, -1) node[anchor=north] {$\vec U_{\ir U}$} ;
\end{circuitikz}

 \subsubsection{Dreieckschaltung}
\begin{circuitikz}
\draw
(2,0.1) node[anchor=south] {U} (2,0) to[L, o-, v_>=$\vec U_{\ir UV}$] (2,-1.5) -- (2.2, -1.5) -- (2.4, -0.2) -- (2.6, -0.2)
(2.6,0.1) node[anchor=south] {V} (2.6, 0) to[L, o-] (2.6, -1.5) -- (2.8, -1.5) -- (3, -0.2) -- (3.2, -0.2)
(3.2,0.1) node[anchor=south] {W} (3.2, 0) to[L, o-] (3.2, -1.5) -- (3.2, -1.7) -- (1, -1.7) -- (1, -0.2) -- (2, -0.2);
\end{circuitikz}

 \subsubsection{Zickzackschaltung}  
\end{multicols*}
\end{document}