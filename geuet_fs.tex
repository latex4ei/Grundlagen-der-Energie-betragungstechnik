\documentclass[fs, footer]{latex4ei}

\begin{document}
\begin{multicols*}{4}
\fstitle{Grundlagen der \\ Energieübertragungstechnik}


\section{Rechnerische Behandlung des Drehstromsystems}

\section{Freileitungen und Kabel}

Widerstandsbelag: $R_{\vartheta}'= \frac{1}{K_{20} Q} \left[ 1+ \alpha ( \vartheta - 20 K) \right]$\\ 

\sectionbox{
\subsection{Freileitungen}

Induktivitätsbelag: $L_b' = (2 \ln \frac{D_{\ir ers}}{r} + \frac 1 2 ) \cdot 10^{-4} \si{\henry \per \kilo \meter}$ \\


Kapazitätsbelag: $C_b ' = \frac{2 \pi \epsilon_0}{\ln \left( \frac{D_{\ir ers}}{r_B \sqrt{1 + \left( \frac{D_{\ir ers}}{2 h}\right)^2 } }\right)} $ \\ 


$D \ll 2h$ $C_b' = \frac{2 \pi \epsilon_0}{\ln \left( \frac{D_{\ir ers}}{ r_B} \right) }$ \\ 


Ohmscher Querleitwert $G'$:
$G_b' = \frac{P_V'}{U_n^2}$

}
\sectionbox{
\subsection{Kabel}

Querschnitt: $Q = (r_a^2 - r_i^2) \pi$ \\ 


Induktivitätsbelag eines Hohlleiters:

$\omega L_{b_{\ir HL}}' = ( 0,96 + 0,051 \frac{r_a - r_i}{r_a})$ für $0 < \frac{r_a - r_i}{r_a} < 0,6$
\\

Kapazitätsbelag: $C_b' = \frac{2 \pi \epsilon}{\ln \left( \frac{R_a}{R-i} \right )}$ \\ 


Ohmscher Querleitwert:
$G_b' = \tan \delta \cdot \omega C_b'$
}
\section{Leitung im stationären und nichtstationären Betrieb}
\sectionbox{
\subsection{Querkompensation}

 $\beta = \sqrt{ L ' C' }$ \\ 
$\vartheta_{\ir nat} = \beta l $

$Z_w = \sqrt{\frac{\omega L'}{\omega C'}}$ \\ 

$\vec Z_l = Z_w \j \sin ( \beta l)$ \\

$\frac{Y_q}{2} = \frac{1}{Z_w} \j \tan \left( \frac{\beta l}{2} \right)$

$P_{\ir nat} = \frac{U_n^2}{Z_w}$

\boxed{
Q_2 = \frac{P_{\ir nat}}{\sin ( \beta l )} \left[ \sqrt{1 - \left( \frac{p_2}{P_{\ir nat}} \sin (\beta l) \right)^2} - \cos (\beta l) \right]}
}
\section{Transformatoren}
\end{multicols*}
\end{document}