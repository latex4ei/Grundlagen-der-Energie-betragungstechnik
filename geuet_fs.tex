
\documentclass[fs, footer]{latex4ei}
\usepackage[european]{circuitikz}
\usepackage{tikz}
\begin{document}
\begin{multicols*}{4}
\fstitle{Grundlagen der \\ Energie- \\ übertragungstechnik}

 \ctikzset{bipoles/length=0.6cm}
\section{Rechnerische Behandlung des Drehstromsystems}

\sectionbox{
	 \subsection{Spannung $u$}
	
	 Augenblickswert:
	 $u(t) = \hat u \cos ( \omega t + \varphi_u) = U \sqrt 2 \cos (\omega t + \varphi_u)$ \\ 

	 Scheitelwert / Amplitude:  $\hat u$ \\ 
 
	 Effektivwert (allg.): $U_{\ir eff} = U = \sqrt{\frac{1}{T} \int \limits_{t_0}^{t_0 + T} u^2(\tau) \diff \tau}$ \\
	 $\ra $ bei sinusförmigen Größen: $U = \frac{\hat u}{\sqrt 2}$ \\ 

	 Phasenwinkel: $\varphi (t) = \omega t + \varphi_u$ mit $\omega = 2 \pi f = \frac{2 \pi}{T}$ \\ 

	 Nullphasenwinkel: $\varphi_u$ \\ 
}
\sectionbox{

	\subsection{Stromstärke $i$ }
	 Augenblickswert:
	 $i(t) = \hat{i} \cos ( \omega t + \varphi_i) = I \sqrt 2 \cos (\omega t + \varphi_i)$ \\ 

	 Scheitelwert / Amplitude:  $\hat i$ \\ 
 
	 Effektivwert (bei sinusförmigen Größen): $I = \frac{\hat i}{\sqrt 2}$ \\ 

	 Phasenwinkel: $\varphi (t) = \omega t + \varphi_i$ mit $\omega = 2 \pi f = \frac{2 \pi}{T}$ \\ 

	 Nullphasenwinkel: $\varphi_i$ \\ 

	 \textbf{Phasenverschiebungswinkel:} \\ 
	 $\varphi = \varphi_{ui} = \varphi_u - \varphi_i$ \\
	 
	  $0 < \varphi \le \pi$: Strom eilt der Spannung nach \\
	 $-\pi \le \varphi < 0$: Strom eilt der Spannung vor
}
\sectionbox{
	 \subsection{Symmetrische Komponenten} 

	 zerlegen eines Dreileiter-Drehstromnetz in unabhängige Systeme (Mit-, Gegen- und Nullsystem) \\


	 Entsymmetrierungsmatrix $\ma T$:

	 $\vect{\vec I_1 \\ \vec I_2 \\ \vec I_3} = \mat{1 & 1 & 1 \\ \vec a^2 & \vec a & 1 \\ \vec a & \vec a^2 & 1} \cdot \vect{\vec I_{(1)1} \\ \vec I_{(2)1} \\ \vec I_{(0)1}}$ \\

	 Symmetrierungsmatrix $\ma S$:

 	$\vect{\vec I_{(1)1} \\ \vec I_{(2)1} \\ \vec I_{(0)1}} = \frac{1}{3} \cdot \mat{1 & \vec a & \vec a^2 \\ 1 & \vec a^2 & \vec a \\ 1 & 1 & 1 } \cdot  \vect{\vec I_1 \\ \vec I_2 \\ \vec I_3} $ \\ \\ \\

 	Mitsystem:

 	\begin{circuitikz}[scale=0.8]
 	\draw (0,0) -- (1,0) to[R, o-o, l^=$\vec{Z_{(1)}}$, i^>=$\vec{I_{(1)}}$ ] (4,0) to[open, v^>=$\vec{U_{(1)}}$, o-o] (4, -1.5) -- (1, -1.5) to[short, o-] (0,-1.5) to[sV, v^<=$\vec{E_{1}}$] (0,0);
 	\end{circuitikz} \\

 	Gegensystem:

 	\begin{circuitikz}[scale=0.8]
 	\draw (0,0) -- (1,0) to[R, o-o, l^=$\vec{Z_{(2)}}$, i^>=$\vec{I_{(2)}}$ ] (4,0) to[open, v^>=$\vec{U_{(2)}}$, o-o] (4, -1.5) -- (1, -1.5) to[short, o-] (0,-1.5) to[short] (0,0);
 	\end{circuitikz} \\

 	Nullsystem:

 	\begin{circuitikz}[scale=0.8]
 	\draw (0,0) -- (1,0) to[R, o-o, l^=$\vec{Z_{(0)}}$, i^>=$\vec{I_{(0)}}$ ] (4,0) to[open, v^>=$\vec{U_{(0)}}$, o-o] (4, -1.5) -- (1, -1.5) to[short, o-] (0,-1.5) to[short] (0,0);
 	\end{circuitikz}

}

\sectionbox{
\subsection{Stern-Dreieck-Umwandlung}

\begin{circuitikz}
 \draw (1,0) to[L, l=$\vec Z_{31}$, o-o, i_>=$\vec I_{13}$] (0, -2) to[L, l=$\vec Z_{23}$, o-o, i_>=$\vec I_{32}$] (2, -2) to[L, l=$\vec Z_{12}$, o-o, i_>=$\vec I_{21}$] (1,0);
 
\draw (4,0) to[L, l=$\vec Z_1$, i_>=$\vec I_1$, o-o] (4, -1) to[L, l=$\vec Z_3$, i^<=$\vec I_3$, o-o] (3, -2);
\draw (4, -1) to[L, l=$\vec Z_2$, i^<=$\vec I_2$, o-o] (5, -2);  
\end{circuitikz}

$\vec Z_1 = \frac{\vec Z_{12} \vec Z_{31}}{\vec Z_{12} + \vec Z_{23} + \vec Z_{31}}$

$\vec Z_2 = \frac{\vec Z_{12} \vec Z_{23}}{\vec Z_{12} + \vec Z_{23} + \vec Z_{31}}$

$\vec Z_3 = \frac{\vec Z_{23} \vec Z_{31}}{\vec Z_{12} + \vec Z_{23} + \vec Z_{31}}$

$\vec I_1 = \vec I_{12} - \vec I_{31}$

$\vec I_2 = \vec I_{23} - \vec I_{12}$

$\vec I_3 = \vec I_{31} - \vec I_{23}$
}
 	\section{Freileitungen und Kabel}



\subsection{Freileitungen}

\sectionbox{
\subsubsection{Durchhang der Freileitungsseile}

\symbolbox{
\begin{tabular}{cc}
 $f(x)$ & Durchhang an der Stelle $x$ \\
$G'$ & Gewichtskraft des Leiterseils pro Längeneinheit \\
$F_H$ & Horizontalzugkraft \\
$h$ & mittlere Höhe der Leiterseile
\end{tabular}
}


Durchhang: $f(x) = f_{\ir max} - \frac{G'}{2 F_h} x^2)$ mit $f_max = \frac{G'}{8 F_H} a^2$


mittlere Höhe: $h = h_{\ir Mast}  - 0,7 \cdot f_{\ir max}$ \\ 

Mindestabstand zum Erdboden (VDE 0210-1):


\symbolbox{
\begin{tabular}{cc}
$D_{\ir el}$ & elektrischer Grundabstand
\end{tabular}
} \\ 

Mindestabstand: $h_{\ir min} = 5 \si{\meter} + D_{\ir el}$ \\ 

Material:

z.B. Al/St 240/40:

Querschnitt des Aluminiumleiters $240 \si{\milli \meter \squared}$, Querschnitt des Stahlleiters $40 \si{\milli \meter \squared} $
}
\sectionbox{
\subsubsection{Bündelleiter}

\symbolbox{
	\begin{tabular}{cc}
	$r_B$ & Ersatzradius \\
	$n$ & Anzahl der Teilleiter \\
	$r$ & Seilradius eines Teilleiters \\
	$r_T$ & Teilkreisradius
	\end{tabular}
}
für einen Bündelleiter mit 4 Teilleitern: $r_T = \sqrt 2 \frac{a}{2}$

Ersatzradius $r_b$: $r_n = \sqrt[n]{n \cdot r \cdot r^{n-1}_{T}} $

\textbf{Abstand zwischen den Außenleitern} \\ 
Ersatzabstand $D_{\ir ers}$ für Einfachsysteme: $D = \sqrt[3]{D_{12} \cdot D_{23} \cdot D_{31}}$ \\ \\
Ersatzabstand für Doppelsysteme mit $\gamma$-Verdrillung: \\
$D =  \sqrt[3]{D_{12} \cdot D_{23} \cdot D_{31} \cdot \frac{D_{12'} \cdot D_{23'} \cdot D_{31'}}{D_{11'} \cdot D_{22'} \cdot D_{33'}}}$

typische Angabe des Werte:

$4 \cdot \frac{21,7}{400} \si{\milli \meter} \Ra a = 400 \si{\milli \meter}, r=\frac{21,7}{2} \si{\milli \meter} $
}
\sectionbox{
\subsubsection{Widerstandsbelag}

Durch Ladungsträgerbewegung bei Stromfluss wird Joule'sche Wärme frei.

\symbolbox{
	\begin{tabular}{cc}
	$Q$ & Querschnitt \\
	$\alpha$ & Temperaturkoeffizient \\
	$\vartheta$ & Celsiustemperatur
	\end{tabular}
}

Widerstand pro Längeneinheit: $R_{20}' =  \frac{1}{\kappa_{20} Q}$ \\
mit Temperaturabhängigkeit: $R_{\vartheta}'= \frac{1}{K_{20} Q} \left[ 1+ \alpha ( \vartheta - 20 K) \right]$\\ 

im Bündelleiter: $R_B = \frac{1}{n} \cdot R$ mit $R$ ist Widerstand des Teilleiters
}
\sectionbox{
\subsubsection{Induktivitätsbelag}

Induktivität einer Leiterschleife mit Hin- und Rückleiter:

$L_{00} = l \frac{\mu_0}{\pi} \cdot (\frac{1}{4} + \ln \frac{d}{r})$ \\

Koppelinduktivitäten:

$M = M_{12} = M_{23} = M_{31} = \frac{\mu_0}{2\pi} l (\frac{1}{4} + \ln \frac{d^2}{rD})$ \\

Betriebsinduktivität:

$L_b \cdot \vec I_1 = L_{00} \cdot \vec I_1 + M \cdot \vec I_2 + M \cdot \vec I_3 = (L_{00} - M) \cdot \vec I_1$ \\

\emphbox{Betriebsinduktivitätsbelag: $L_b' = (2 \ln \frac{D_{\ir ers}}{r} + \frac 1 2 ) \cdot 10^{-4} \si{\henry \per \kilo \meter}$} \\
für Bündelleiter: $L_b' = (2 \ln \frac{D_{\ir ers}}{r_B} + \frac{1}{2 n}) \cdot 10^{-4} \si{\henry \per \kilo \meter}$ 
}

\sectionbox{
	\subsubsection{Betriebsimpedanzen}

	Einfachseil: $\vec Z_b' = \vec Z_{(1)}' = R' + \j \omega \frac{\mu_0}{2\pi} (\ln \frac{D}{r} + \frac{1}{4})$ \\

	Bündelleiter: $\vec Z_b' = \vec Z_{(1)}' = \frac{R'}{n} + \j \omega \frac{\mu_0}{2\pi} (\ln \frac{D}{r_B} + \frac{1}{4n})$ \\

	Typische Werte (Al/St 240/40) bei $50 \si{\hertz}$: \\
	\tablebox{
	\begin{tabular*}{\columnwidth}{@{\extracolsep\fill}lll@{}} \ctrule
		$110 \si{\kilo \volt} $ & Einfachseil & $\vec Z_b' = (0,12 + \j 0,4) \si{\ohm \per \kilo \meter} $ \\
		$220 \si{\kilo \volt} $ & 2er-Bündel & $\vec Z_b' = (0,06 + \j 0,3) \si{\ohm \per \kilo \meter} $ \\
		$380 \si{\kilo \volt} $ & 4er-Bündel & $\vec Z_b' = (0,03 + \j 0,25) \si{\ohm \per \kilo \meter} $ \\
		\cbrule
	\end{tabular*}
	}

} 

\sectionbox{
	
	\subsubsection{Nullimpedanz}

	Erdstromtiefe: $\delta = \frac{1,85}{\sqrt{\mu_0 \cdot \frac{1}{\rho} \cdot \omega} }$ 

	Nullimpedanz für Einfachseile:

	$\vec Z_{(0)} = \frac{\vec U_{(0)}}{\vec I_{(0)}} = R' + 3 \omega \frac{\mu_0}{8}+ \j \omega \frac{\mu_0}{2\pi} (3 \ln \frac{\delta}{\sqrt[3]{r \cdot D^2}} + \frac{1}{4})$ \\

	Nullimpedanz für Bündelleiter:

	$\vec Z_{(0)} = \frac{\vec U_{(0)}}{\vec I_{(0)}} = \frac{R'}{n} + 3 \omega \frac{\mu_0}{8}+ \j \omega \frac{\mu_0}{2\pi} (3 \ln \frac{\delta}{\sqrt[3]{r \cdot D^2}} + \frac{1}{4n})$
} 
\sectionbox{
\subsubsection{Kapazitätsbelag}
 $C_b ' = \frac{2 \pi \epsilon_0}{\ln \left( \frac{D_{\ir ers}}{r_B \sqrt{1 + \left( \frac{D_{\ir ers}}{2 h}\right)^2 } }\right)} $ \\ 


für $D \ll 2h$: $C_b' = \frac{2 \pi \epsilon_0}{\ln \left( \frac{D_{\ir ers}}{ r_B} \right) }$ \\ 

}

\sectionbox{
	\subsubsection{Nullkapazität}

	$C_{(0)}' = \frac{2\pi \epsilon_0}{3 \ln \frac{2h}{\sqrt[3]{rD^2}}}$ 

} 
\sectionbox{
\subsubsection{Ohmscher Querleitwert $G'$}
Verluste durch Ableitströme über Isolatoren und durch Koronaverluste. \\

spezifische Arbeitsverluste: $P_V' = 3 \cdot \left( \frac{U_n}{\sqrt{3}} \right)^2 \cdot G_b' = U_n^2 \cdot G_b'$ \\

Betriebsableitbelag: $G_b' = \frac{P_V'}{U_n^2}$

}
\sectionbox{
\subsection{Kabel}


Querschnitt: $Q = (r_a^2 - r_i^2) \pi$ \\ 


Eindringtiefe: $\delta = \sqrt{\frac{2}{\omega \mu \kappa}}$

\subsubsection{Widerstandsbelag $R'$}

Widerstand pro Längeneinheit: $R_{20}' =  \frac{1}{\kappa_{20} Q}$ \\
mit Temperaturabhängigkeit: $R_{\vartheta}'= \frac{1}{K_{20} Q} \left[ 1+ \alpha ( \vartheta - 20 K) \right]$\\ 


\subsubsection*{Stromverdrängung / Skineffekt}
für dicke massive Leiter. Erhöhung des ohmschen Widerstandes des Leiters.  
\symbolbox{
	\begin{tabular}{cc}
	$y_s$ &  Skineffekt-Korrekturfaktor \\
	$R_{~}$ & Wechselstromwiderstand \\
	$R_{=}$ & Gleichstromwiderstand \\
	$f$ & Frequenz
	\end{tabular}
} 

$\frac{R_{\approx}}{R_{=}} = 1 + y_s$

$y_s = \frac{x_s^4}{192 + 0,8 \cdot x_s^4}$

$x_s = \sqrt{\frac{2 \cdot \mu \cdot f \cdot k_s}{R_{=}'}} $ mit $k_s = \begin{cases}
1 & \text{für Rundleiter} \\ 0,5 & \text{für Segmentleiter}
\end{cases}$
\subsubsection{Induktivitätsbelag $L'$}
Betriebsinduktivitäsbelag: $L_b' = \frac{L_B}{l} = \left( 2 \ln \frac{D}{r}+ \frac{1}{2} \right) \cdot 10^{-4} \si{\henry \per \kilo \meter}$ \\

Betriebsreaktanzbelag: $X_b ' = \omega	L_b ' $ \\ 

Induktivitätsbelag eines Hohlleiters:

$\omega L_{b_{\ir HL}}' = \omega L_{b}' \cdot ( 0,96 + 0,051 \frac{r_a - r_i}{r_a})$ für $0 < \frac{r_a - r_i}{r_a} < 0,6$
\\
}
\sectionbox{
\subsubsection{Kapazitätsbelag $C'$}

Radialfeldkabel: \\
$C_b' = \frac{2 \pi \epsilon}{\ln \left( \frac{R_a}{R_i} \right )}$ \\ 
$C_L' = 0$

Gürtelkabel: \\
$C_b' = \frac{2 \pi \epsilon_0 \epsilon_r}{\ln{\sqrt{\frac{3c^2 (R_a^2 - c^2)^3}{R_i^2 (R_a^6 - c^6)} } } }$

}
\sectionbox{
\subsubsection{Ohmscher Querleitwert}
$G_b' = \tan \delta \cdot \omega C_b'$
}

\sectionbox{
\subsubsection{typische Werte}
\;
	\tablebox{
	\begin{tabular*}{\columnwidth}{@{\extracolsep\fill}lll@{}} \ctrule
		Symbol & Einheit & typische Werte \\ \cmrule
		$R'$ & $[\si{\ohm \per \kilo \meter}]$ & $0,01 \ldots 0,05$  \\
		$X' = \omega L_b'$ & $[\si{\ohm \per \kilo \meter}]$& $0,1 \ldots 0,2 $ \\
		$Y_b' = \omega C_b'$&  $[\si{\micro \siemens \per \kilo \meter} ]$& $50 \ldots 100 $ \\
		$Z_w $ & $[\si{\ohm} ]$ &$32 \ldots 63 $ \\ 
		\cbrule
	\end{tabular*}
	}
}

\section{Leitung im stationären und nichtstationären Betrieb}

\sectionbox{
	\subsection{Vereinfachte Leitungsbetrachtung}

	vernachlässigen von $G_b$ und $Y_b$

	\begin{circuitikz}
	 \draw(0,0) to[R, o-, i_>=$\vec{I}$,  l=$R' \cdot l$] (2,0) to[L, l=$\j \omega L_b' l$, i_>=$\vec I$] (4,0) to[open, v^>=$\vec U_2$, -o] (4, -1) -- (0,-1) to[open, o-o, v^<=$\vec U_1$] (0,0);
	 \end{circuitikz} 

	 Längsspannungsfall: $\Delta U = R \cdot I_w + \omega L_b \cdot I_b$

	 Querspannungsfall: $\delta U = \omega L_b \cdot I_w - R \cdot I_b$
}


\sectionbox{
	\subsection{Leitungsgleichung für einphasige Leitung}


	$\frac{\partial^2 u}{\partial x^2} = L' \cdot C' \cdot \frac{\partial^2 u}{\partial t^2} + (R' \cdot C' + L' \cdot G') \cdot \frac{\partial u}{\partial t}+ R' \cdot G' \cdot u$

	$\frac{\partial^2 i}{\partial x^2} = L' \cdot C' \cdot \frac{\partial^2 i}{\partial t^2} + (R' \cdot C' + L' \cdot G') \cdot \frac{\partial i}{\partial t}+ R' \cdot G' \cdot i$ \\ 

	Übertragungsmaß:

	$\vec \gamma = \alpha + \j \cdot \beta = \sqrt{(R'(\omega) + \j \omega L'(\omega)) \cdot (G'(\omega) + \j \omega C'(\omega))} $ 
} 
\sectionbox{
\subsection{Leitungsgleichungen für Drehstromleitungen} 

Übertragungsmaß $\vec \gamma = \alpha + \j \beta = \sqrt{(R' + \j \omega L') \cdot (G' + \j \omega C')} $

Betriebswellenimpendanz $\vec Z_W = \sqrt{\frac{R' + \j \omega L'}{G' + \j \omega C'}}  $

Zweitorgleichung der verlustbehafteten Leitung:

$\vect{\vec U_1 \\ \vec I_1} = \mat{\cosh (\vec \gamma \cdot l) & \vec Z_w \cdot \sinh(\vec \gamma \cdot l) \\ \frac{1}{\vec Z_w} \cdot \sinh(\vec \gamma \cdot l) & \cosh(\vec \gamma \cdot l)} \cdot \vect{\vec U_2 \\ \vec I_2}$ \\  \\

\textbf{$\pi$ - Ersatzschaltbild}

\begin{circuitikz}
\draw (0,0) to[short, o-*, i^>=$\vec I_1$] (1,0) to[R, l=$\vec Z_l$] (3,0) to[short, *-o, i^>=$\vec I_2$] (4,0) to[open,v^>=$\vec U_2$] (4, -1.5) to[short, o-*] (3, -1.5) -- (1, -1.5) to[short, *-o] (0,-1.5) to[open, v^<=$\vec U_1$] (0,0);
\draw (1,0) to[R, l=$\frac{\vec Y_q}{2}$] (1, -1.5);
\draw (3,0) to[R, l=$\frac{\vec Y_q}{2}$] (3, -1.5);
\end{circuitikz}


Längsimpedanz $\vec Z_l = \vec Z_w \cdot \sinh(\vec \gamma \cdot l)$ 

Queradmittanz $\frac{\vec Y_q}{2} = \frac{1}{\vec Z_w} \cdot \tanh(\vec \gamma \cdot \frac{l}{2})$


}
\sectionbox{
\subsection{verlustlose Fernleitung}
Phasenmaß: $\beta = \omega \sqrt{ L' C' }$ \quad $\left[\si{1 \per \kilo \meter} \right]$ \\ 
$\vartheta_{\ir nat} = \beta l $

Wellenwiderstand für verlustlose Leitungen: $Z_W = \sqrt{\frac{\omega L'}{\omega C'}}$ \\ 

Eine Leitung gilt als elektrisch kurz für $l \le 200 \si{\kilo \meter} $ (Freileitung) bzw. $l \le 100 \si{\kilo \meter} $ (Kabel) \\ 

Längsimpedanz (elektrisch lang): $\vec Z_l = Z_w \j \sin ( \beta l)$ \\
Längsimpedanz (elektrisch kurz): $\vec Z_l = \j \omega L' \cdot l$ \\
Queradmittanz (elektrisch lang): $\frac{\vec Y_q}{2} = \frac{1}{Z_w} \j \tan \left( \frac{\beta l}{2} \right)$ \\
Queradmittanz (elektrisch kurz): $\frac{\vec Y_q}{2} = \j \omega C' \cdot \frac{l}{2}$ \\




\emphbox{
Zweitorgleichung:
$\mat{ \vec U_1  \\ \vec I_1 } = \mat{ \cos(\beta l) & Z_w \j \sin ( \beta l) \\ \frac{1}{Z_w} \j \sin (\beta l) & \cos ( \beta l)} \mat{\vec U_2  \\ \vec I_2 }$
}
} 

\sectionbox{
	\subsection{Eingangsimpedanz}
	natürliche Leistung: $P_{\ir nat} = 3 \cdot \frac{U_1^2}{Z_W} = \frac{U_n^2}{Z_w}$ \\
	gibt einen Anhaltswert für die Übertragungsfähigkeit einer Leitung im symmetrischen Drehstromsystem.\\

	\subsubsection{Kurzschluss $\vec Z_2 = 0$}
	$\vec Z_1 = Z_W \cdot \j \tan (\beta l)$ \\

	\subsubsection{Leerlauf $\vec I_2 = 0$}
	$\vec Z_1 = Z_W \cdot \frac{1}{\j \tan ( \beta l)}$ \\ 
	$\vec U_1 = \cos (\beta l) \cdot \vec U_2$ \\

	\subsubsection{Abschluss mit der Wellenimpedanz $\vec{Z_2} = Z_W$} 
	Betrieb mit natürlicher Leistung:
	\begin{itemize}
		\item $\abs{\vec U_1} = \abs{\vec U_2}$
		\item $\angle (U, I) = 0$
		\item Phasendrehung von $U$ und $I$ um $\beta l$
	\end{itemize}
 }

\sectionbox{
\subsection{Blindleistungskompensation}
\symbolbox{
	\begin{tabular}{cc}
	$k_q$ & Querkompensationsgrad \\
	$k_l$ & Längskompensationsgrad \\
	$C_w'$ & wirksamer Kapazitätsbelag \\
	$L_w'$ & wirksamer Induktivitätsbelag
	\end{tabular}
}

wirksamer Wellenwiderstand $Z_{W k} = \sqrt{\frac{L_w'}{C_w'}}  = \sqrt{\frac{L'}{C'}} \cdot \sqrt{\frac{1 - k_l}{1 - k_q}}	  $

$\frac{P_{\ir nat k}}{P_{\ir nat}} = \sqrt{\frac{1-k_q}{1-k_l}} $

$\beta_k = \omega \sqrt{L_w' \cdot C_w'} = \omega \sqrt{L' \cdot C'} \cdot \sqrt {(1-k_l)\cdot (1-k_q)}$

}
\sectionbox{
\subsection{Querkompensation}

wirksamer Kapazitätsbelag: $C_w' = C' \cdot (1-k_q) \Ra (1 - k_q) = \frac{C_w'}{C'}$ \\

\emphbox{
Kompensationsblindleistung am Leitungsende\\ 
$Q_2 = \frac{P_{\ir nat}}{\sin ( \beta l )} \left[ \sqrt{1 - \left( \frac{P_2}{P_{\ir nat}} \sin (\beta l) \right)^2} - \cos (\beta l) \right]$
} \\

falls Leitung mit Kompensationsinduktivität $L_k$ abgeschlossen ist:

$\vec I_2 = \vec I_k = \frac{\vec U_2}{\j \omega L_k}$

Kompensationsreaktanz: $X_k = \omega L_k = \frac{Z_w \cdot \sin (\beta l)}{1 - \cos (\beta l)}$
}
\sectionbox{
\subsection{Längskompensation}

wirksamer Induktivitätsbelag: $L_w' = L' (1 - k_l)$

Längskompensationsgrad: $k_l = 1 - \frac{L_w'}{L'} = \frac{\omega L' \cdot l - (\omega L' \cdot l - \frac{1}{\omega C_k})}{\omega l' \cdot l}$ \\ 

Faustregel für die optimale Anzahl der Kondensatorbatterien: \\
$0 < k_l \le 0,5 \Ra n = 1 \Ra X_k = k_l \cdot 2 Z_w \sin (\beta \frac{l}{2})$ \\
$0,5 < k_l \le 0,67 \Ra n = 2 \Ra X_k = k_l \cdot \frac{3}{2} Z_w \sin (\beta \frac{l}{3})$ \\
$0,67 < k_l \le 0,75 \Ra n = 3 \Ra X_k = k_l \cdot \frac{4}{3} Z_w \sin (\beta \frac{l}{4})$ \\ \\ 

Komensationsblindleistung: $Q_K = 3 \cdot X_K \cdot I_K^2$

Leistungswinkel der kompensierten Leitung: $\vartheta_k = \beta_k l$

Grenzwinkel für Stabilität der Leitung: $\vartheta_{\ir Grenz} = 42^\circ$
$\vartheta = (\vartheta_M)_{\ir grenz} - (\vartheta_M + \vartheta_T)$ mit Transformatorwinkel $\vartheta_T \approx 3,5^\circ$, $\vartheta_M \approx 5,5^\circ$
}

\sectionbox{
	\subsection{Übertragungsfähigkeit von Freileitungen}

	\tablebox{
	\begin{tabular*}{\columnwidth}{@{\extracolsep\fill}cccccc@{}} \ctrule
	$U_n$ & Leiter & $Z_W  $ & $P_{\ir nat}  $ & $S_{\ir th}$ \\ 
	$[\si{\volt} ]$& & $[\si{\ohm}]$ & $[\si{\mega \watt}] $ & $[\si{\mega \volt \ampere}]$   \\ \cmrule 
	$10 \si{\kilo \volt}$ & Al/St 50/8 & 330 & 0,3 & 3 \\
	$30 \si{\kilo \volt}$ & Al/St 95/15 & 360 & 2,5 & 15  \\
	$110 \si{\kilo \volt}$ & Al/St 240/40 & 365 & 33 & 123  \\
	$380 \si{\kilo \volt}$ & Al/St $4 \cdot 240/40$ & 240 & 600 & 1700  \\
	$380 \si{\kilo \volt}$ & Al/St $4 \cdot 550/70$ & 240 & 600 & 2633 \\
	$735 \si{\kilo \volt}$ & Al/St $4 \cdot 680/85$ & 260 & 2080 & 5860  \\ \cbrule
	\end{tabular*}
	} 
}
\sectionbox{
\subsection{Wanderwellen}
}

\section{Transformatoren}

\sectionbox{
\subsection{Zweiwicklungstransformator}

 \ctikzset{bipoles/length=0.7cm}
\begin{circuitikz}[ scale =0.5]
\draw
(0,0) to[R, o-, i>^=$\vec I_1$, l=$R_1$] (3,0) to[L, -*, l=$\j X_{\sigma 1}$] (5,0 ) to[short, -*] (6,0) to[L, l=$\j X_{\sigma 2}^{\times}$] (8,0)
 to[R,l=$R_2^\times$, i^>=$\vec I_2^\times$, -o] (12,0) to[open, v^>=$\vec U_2^\times$] (12, -4) -- (6, -4) to[short, *-*] (5, -4) to[short, -o] (0, -4) to[open, v^<=$\vec U_1$] (0,0)
(5,0) to[R, l_=$R_{\ir Fe}$, i>_=$\vec I_{\ir h}$] (5, -4)
(6,0) to[L, l=$\j X_{\ir h}$, i>=$\vec I_{\ir Fe}$] (6, -4)
 ;
\end{circuitikz}

Übersetzung: $\textrm{ü} = \frac{w_1}{w_2}$ \\

$\vec I_2^\times = \frac{\vec I_2}{\textrm{ü}}$ mit $w_1, w_2$ sind Windungszahlen \\
$\vec U_2^\times = \textrm{ü} \vec U_2$  \\
$\vec Z^\times_2 = \textrm{ü}^2 \vec Z_2$ \\
$\vec U_2^\times \cdot \vec I_2^\times = \textrm{ü}  \vec U_2 \cdot \frac{\vec I_2}{\textrm{ü}} = \vec U_2 \cdot \vec I_2$


Bemessungsstrom: $\vec I_r = \frac{S_{rT}}{\sqrt{3} U_{rT}}$

Kupferverluste: $P_{\ir cu} = 3 U_{R_k} I_1 = 3 R_k J_1^2$
}

\sectionbox{
\subsubsection{Leerlauf}

Leerlaufstrom: $\vec I_0 = \vec I_{\ir Fe} + \vec I_h$

Hauptreaktanz: $X_h = \Im{\frac{\vec U_{r1}}{\vec I_h}}$ 

Eisenverluste (im Einphasentransformator): $R_{\ir Fe} = \frac{U_{r1}^2}{P_0} = \textrm{ü}^2 \cdot \frac{U^2_{r2}}{P_0}$
}
\sectionbox{
\subsubsection{Kurzschluss}

\begin{circuitikz}
\draw
(0,0) to[R, o-, i>^=$\vec I_1$, l=$R_k$] (1.5,0) to[L, l=$\j X_k$, i^>=$\vec I_2^\times$] (3,0) -- (3,-1) to[short, -o] (0, -1) to[open, v^<=$\vec U_1$] (0,0)
;
\end{circuitikz} \\ 

Kurzsschlussimpedanz: \\
$\vec Z_k = \frac{\vec U_k}{\vec I_{r1}} = (R_1 + R_2^\times) + \j (X_{\sigma1} + \X_{\sigma2}^\times) = R_k + \j X_k$ \\ \\ 

relative Kurzschlussspannung: \\
$u_k = \frac{U_k}{U_{r1}} = \frac{I_{r1} \cdot X_k}{U_{r1}} \cdot \left( \frac{U_{r1}}{U_{r1}} \right) = \frac{S_r \cdot X_k}{U_{r1}^2}$ 


Bezogener Spannungsfall: $u_x = \frac{X_k S_{rT}}{U_{r1}^2}$
}
\sectionbox{
\subsubsection{Bemessungsstrom}

\begin{circuitikz}
\draw(0,0) to[R, o-, l=$R_k$, i>^=$\vec{I_1}$](1.5,0) to[L, -o, l=$\j X_k$, i^>=$\vec I_2^\times$] (3,0) to[open, v^>=$\vec{U_2^\times}$] (3,-1)
to[short, o-o] (0,-1)to[open, v^<=$\vec{U_1}$] (0,0);
\end{circuitikz}

$X_h \ra \infty$ \quad $R_{\ir Fe} \ra \infty$ \\

$R_k = R_1 + R_2^\times$ \quad $X_k = X_{\sigma1} + X_{\sigma2}^\times$ \quad $\vec Z_k = R_k + \j X_k$
}
\sectionbox{
\subsubsection{Übersetzung \textrm{ü}}

Übersetzung: $\textrm{ü} = \frac{w_1}{w_2}$ mit $w_1 :=$ Primärwicklung, $w_2 :=$ Sekundärwicklung \\

Leerlaufübersetzung $\textrm{ü}_0 = \frac{\vec U_1}{\vec U_2} \approx \frac{w_1}{w_2}$

}
\sectionbox{
\subsection{Drehstromtransformator}

Kurzschlussspannung: $u_k = \frac{U_{kT}}{U_{rT}} = \frac{\frac{U_{kT}}{\sqrt 3}}{\frac{U_{rT}}{\sqrt 3}} = \frac{X_k \cdot I_{rT}}{\frac{U_{rT}}{\sqrt 3}}$

Bemessungsleistung: $S_{rT} = \sqrt{3} \cdot U_{rT} \cdot I_{rT}$

Kurzschlussreaktanz: $X_k = \frac{u_k \cdot U_{rT}^2}{S_{rT}}$ 

$X_{k(Y)} = \frac{X_{k(\Delta)}}{3}$ \\

Bezogene Kurzschlussspannung: $\vec u_k = u_R + \j u_x$ \\

Typische Werte: \\ 
	\tablebox{
		\begin{tabular*}{\columnwidth}{@{\extracolsep\fill}llll@{}} \ctrule
			Anwendung & $S_r / MVA$ & $u_R / \%$ & $u_x / \%$ \\ \cmrule
			Niederspannung & $0,25 \ldots 1,6$ & $1,8 \ldots 1,0$ & $5,8$ \\
			Mittelspannung & $2,5 \ldots 25$ & $1,0 \ldots 0,5$ & $7 \ldots 8,5$ \\
			Hochspannung $110 \si{\kilo \volt} $ & $16 \ldots 63$ & $0,7 \ldots 0,6$ & 12 \\
			Hochspannung $220 \si{\kilo \volt} $ & $100 \ldots 400$ & $0,5 \ldots 0,3$ & $12 \ldots 14$ \\
			Hochspannung $380 \si{\kilo \volt} $ & $630 \ldots 1500$ & $0,2$ & $13 \ldots 16$ \\
			\cbrule
		\end{tabular*}
	}
}
\sectionbox{
	\subsection{Wicklungsverschaltung}

	\subsubsection{Sternschaltung}
	\begin{circuitikz} [scale=0.5]
		\draw(0,0) node[anchor=south] {U} to[L, v_>=$\vec U_{\ir U}$, o-] (0, -3) -- (1,-3) to[L, -o] (1,0) node[anchor=south] {V}
		(1,-3) -- (2, -3) to[L, -o] (2, 0)  node[anchor=south] {W}
		(3,0) to[short, -o] (3,0) node[anchor=south] {N};
		\draw[dashed] (2, -3) -- (3, -3) to[short] (3,-0.1);
	\end{circuitikz} \quad \quad
	\begin{circuitikz}
		\draw[->] (0.8, -1) -- (0, 0) node[anchor=west] {$\vec{U_{\ir{W}}}$};
		\draw[->](0.8, -1) -- (0, -2) node[anchor=west] {$\vec{U_{\ir{V}}}$};
		\draw[->] (0.8, -1) -- (1.8, -1) node[anchor=north] {$\vec{U_{\ir{U}}}$};
	\end{circuitikz}

	\subsubsection{Dreieckschaltung}
	\begin{circuitikz}
		\draw(2,0.1) node[anchor=south] {U} (2,0) to[L, o-, v_>=$\vec{U_{\ir{UV}}}$] (2,-1.5) -- (2.2, -1.5) -- (2.4, -0.2) -- (2.6, -0.2)
		(2.6,0.1) node[anchor=south] {V} (2.6, 0) to[L, o-] (2.6, -1.5) -- (2.8, -1.5) -- (3, -0.2) -- (3.2, -0.2)
		(3.2,0.1) node[anchor=south] {W} (3.2, 0) to[L, o-] (3.2, -1.5) -- (3.2, -1.7) -- (1, -1.7) -- (1, -0.2) -- (2, -0.2);
	\end{circuitikz}

	 \subsubsection{Zickzackschaltung}  

 }

 \section{Kurzschlussstromberechnung}

 \symbolbox{
 \begin{tabular}{cc}
 $I_k''$ & Anfangskurzschlusswechselstrom \\
 $I_k$ & Dauerkurzschlussstrom \\
 $I_a = I_b$ & Ausschaltwechselstrom \\
 $i_p$ & Stoßkurzschlussstrom
  \end{tabular}
 } 
\sectionbox{	
\subsection{Allgemeines}
 Kurzschluss ist generatornah falls gilt: $I_k'' \ge 2 I_{rG}$ 

\subsubsection{generatorfern}
AC Anteil mit konst. Amplitude + auf Null abklingender aperiodischer DC Anteil 

es gilt: $I_k'' = I_b = I_k$

\subsubsection{generatornah}
AC Anteil mit abklingender Amplitude + auf Null abklingender aperiodischer DC Anteil
}
\sectionbox{
\subsection{Dreipoliger Kurzschluss}

unvermaschtes Netz: $i_p = \kappa \sqrt 2 \cdot I_k''$

vemaschtes Netz:
$\kappa = 1,02 + 0.98 e^{- R/X}$

\emphbox{$I_k'' = \frac{c \cdot \frac{U_{nN}}{\sqrt{3}}}{Z_k}$ }

$i_p = 1,15 \cdot \kappa \sqrt 2 \cdot I_k''$

\subsubsection{Symmetrischer Ausschaltwechselstrom}

Geilanteil: $I_{dc} = \sqrt 2 I_k'' e^{- 2 \pi f t R/X}$

Generatorfern: $I_b = I_k''$

Generatornah: $I_b = \mu \cdot I_k''$ mit
$\mu $ siehe S. 177
}
\sectionbox{
\subsection{Ersatzschaltungen und Ersatzimpedanzen}

\subsubsection{Synchrongenerator}
$X_d'' = x_d'' \frac{U_{rG}^2}{S_{rG}}$

\subsubsection{Netzeinspeisung}
$Z_Q = \frac{c \cdot U_{nQ}}{\sqrt 3 \cdot I_{kQ}''}$

$X_Q = \frac{Z_Q}{\sqrt{1 + (R_Q  / X_Q)^2} }$

\subsubsection{Transformator}

$Z_T = u_{kr} \cdot \frac{U_{rT}^2}{S_{rT}}$

$R_T = u_{Rr} \cdot \frac{U_{rT}^2}{S_{rT}} = \frac{P_{krT}}{3 I_{rT}^2}$

$X_T = \sqrt{Z_T^2 - R_T^2} $ \\
}

\sectionbox{
\subsection{unsymmetrische Kurzschlüsse}

\subsubsection{Zweipoliger Kurzschluss ohne Erdberührung}

$I_{k2}'' = \frac{c \cdot U_n}{2 \abs{\vec Z_{(1)}}} = \frac{\sqrt{3}}{2} I_k''$

$i_{p2} = \frac{\sqrt 3}{2} i_p = \kappa \cdot \sqrt 2 \cdot I_{k2}''$

\subsubsection{Zweipoliger Kurzschluss mit Erdberührung}

$I_{kE2E}'' = \frac{\sqrt 3 \cdot c \cdot U_n}{\abs{\vec Z_{(1)} + 2 \vec Z_{(0)}}}$

$i_{p2E} = \kappa \cdot \sqrt 2 \cdot I_{kE2E}''$

\subsubsection{Einpoliger Erdschluss}

$I_{k1}'' = \frac{\sqrt 3 \cdot c \cdot U_n}{\abs{2 \vec Z_{(1)} + \vec Z_{(0)}}} $

$i_{p1} = \kappa \cdot \sqrt 2 \cdot I_{k1}''$
}
\section{Synchrongenerator}

\sectionbox{
	\subsection{stationärerer Betrieb}

	\symbolbox{
		\begin{tabular}{cc}
		$U$ & Klemmenspannung \\
		$I$ & Strangstrom \\
		$U_p$ & Polradspannung
		\end{tabular}
	} 

	$\vec U = \vec U_p - (R + \j X_d) \cdot \vec I$

	mit $X_d = \omega \cdot (L_h + L_\sigma)$ 

	\begin{circuitikz}
		\draw (0,0) to[L, l=$\j X_h$, -*] (1,0) to[L, l=$\j X_\sigma$] (2,0) to[R, l=$R$, -o, i_>=$\vec I_1$] (4,0) to[open, v>=$\vec U$, o-o] (4, -1) -- (1, -1) to[short, *-] (0, -1);
		\draw (0,0) to[sV=$\vec U_p$] (0,-1);
		\draw (1,0) to[open, v^>=$\vec E$] (1,-1);
	\end{circuitikz}

	bezogene synchrone Reaktanz $x_d = \frac{X_d \cdot I_{rG}}{U_{rG} / \sqrt 3} \approx 2,0$

	synchrone Reaktanz: $X_d = \frac{x_d \cdot U_{rG}^2}{S_{rG}}$

	$X_d \cdot I_w = U_p \sin \theta_M$
	mit $\theta_M$ ist elektrischer Polradwinkel \\

	abgegebene Wirkleistung: $P = 3 \cdot U \cdot I_w$

	\textbf{übererregter Betrieb} $\abs{\vec{U_p} } > \abs{\vec{U} } $
	Maschine gibt induktive Blindleistung ab (wirkt wie Kapazität)

	$\vec I = I_w - \j I_b \Ra Q > 0$

	\textbf{untererregter Betrieb} $\abs{\vec{U_p} } < \abs{\vec{U} } $
	Maschine nimmt induktive Blindleistung auf (wirkt wie Induktivität)

	$\vec I = I_w + \j I_b \Ra Q < 0$ 

}
\section{Sternpunktbehandlung}
\sectionbox{
\subsection{Allgemeines}
\symbolbox{
	\begin{tabular}{cc}
	$U_{\ir LE(F)}$ & Spannung d. fehlerh. Außenleiter bei Erdberührung\\
	$U_{\ir b (F)}$ & Außenleiterspannung wenn kein Fehler wäre \\
	$I_{\ir CE}$ & kapazitiver Erdschlussstrom 
	\end{tabular}
} 

Erdfehlerfaktor (Wirksamkeit der Sternpunkterdung):
$\delta = \frac{U_{\ir LE (F)}}{U_{\ir b (F))}/ \sqrt 3}$ 

Netz ist wirksam geerdet falls $\delta \le 1,4$

Bei wirksamer Erdung hohe Erdkurzschlusströme aber geringe betriebsfrequente und transiente Überspannung
}
\sectionbox{
\subsection{Netz mit isoliertem (freiem) Sternpunkt} 

Bei einpoligem Erdschluss vergrößern sich die Beträge der Leiter-Erde Spnnungen der gesunden Leiter um $\sqrt 3$ $\Ra \delta = \sqrt 3$

Fehlerstrom (bei Erdschluss in L1):\\
 $\vec I_1 = \vec I_{\ir CE} = \j \sqrt 3 \cdot 1,1 \cdot U_{n} \omega C_{E}$
}
\sectionbox{
 \subsection{Netz mit Erdschlusskompensation}

 wirksamer Nullleitwert: $\vec Y_{(0)} = \frac{1}{\j 3 X_{\ir EL}} + \j \omega C_{\ir E}$

Fehlerstrom: $\vec I_{E (F)} = 3 \cdot \vec I_{(0)} = 3 \cdot \frac{U_b}{\sqrt 3} \j \cdot (\omega C_E - \frac{1}{3 X_{EL}})$

Strom durch die Erdlöschspule: $\vec I_{\ir Sp} = \frac{U_b / \sqrt 3}{\j X_{\ir EL}}$
}

\sectionbox{
	\subsection{Netz mit niederohmiger Sternpunkterdung}

	meist in Freileitungsnetzen mit $U_n \ge 220 \si{\kilo \volt} $ bzw. Kabelnetzen mit $U_n \ge 110 \si{\kilo \volt}$

	Erdfehlerfaktor: $\delta = \abs{\vec a^2 + \frac{\vec Z_{(1)}  - \vec Z_{(0)}}{2 \vec Z_{(1)} + \vec Z_{(0)}}} $ 
}
\end{multicols*}
\end{document}