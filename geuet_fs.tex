\documentclass[fs, footer]{latex4ei}

\begin{document}
\begin{multicols*}{4}
\fstitle{Grundlagen der \\ Energieübertragungstechnik}


\section{Rechnerische Behandlung des Drehstromsystems}

\section{Freileitungen und Kabel}


\sectionbox{
\subsection{Freileitungen}


\subsubsection{Durchhang der Freileitungsseile}

\symbolbox{
\begin{tabular}{cc}
 $f(x)$ & Durchhang an der Stelle $x$ \\
$G'$ & Gewichtskraft des Leiterseils pro Längeneinheit \\
$F_H$ & Horizontalzugkraft \\
$h$ & mittlere Höhe der Leiterseile
\end{tabular}
}


Durchhang: $f(x) = f_{\ir max} - \frac{G'}{2 F_h} x^2)$ mit $f_max = \frac{G'}{8 F_H} a^2$


mittlere Höhe: $h = h_{\ir Mast}  - 0,7 \cdot f_{\ir max}$ \\ 

Mindestabstand zum Erdboden (VDE 0210-1):


\symbolbox{
\begin{tabular}{cc}
$D_{\ir el}$ & elektrischer Grundabstand
\end{tabular}
} \\ 

Mindestabstand: $h_{\ir min} = 5 \si{\meter} + D_{\ir el}$ \\ 



\textbf{Bündelleiter} \\
Ersatzradius $r_b$: $r_n = \sqrt[n]{\cdot r \cdot r^{n-1}_{T}} $\\ \\ 

\textbf{Abstand zwischen den Außenleitern} \\ 
Ersatzabstand $D$ : $D = \sqrt[3]{D_{12} \cdot D_{23} \cdot D_{31}}$ \\ \\


\subsubsection{Widerstandsbelag}

\symbolbox{
	\begin{tabular}{cc}
	$Q$ & Querschnitt \\
	$\alpha$ & Temperaturkoeffizient \\
	$\vartheta$ & Celsiustemperatur
	\end{tabular}
}

Widerstand pro Längeneinheit: $R_{20}' =  \frac{1}{\kappa_{20} Q}$ \\
mit Temperaturabhängigkeit: $R_{\vartheta}'= \frac{1}{K_{20} Q} \left[ 1+ \alpha ( \vartheta - 20 K) \right]$\\ 

im Bündelleiter: $R_B = \frac{1}{n} \cdot R$ mit $R$ ist Widerstand des Teilleiters

\subsubsection{Induktivitätsbelag}
Betriebsinduktivitätsbelag: $L_b' = (2 \ln \frac{D_{\ir ers}}{r} + \frac 1 2 ) \cdot 10^{-4} \si{\henry \per \kilo \meter}$ \\


\subsubsection{Kapazitätsbelag}
 $C_b ' = \frac{2 \pi \epsilon_0}{\ln \left( \frac{D_{\ir ers}}{r_B \sqrt{1 + \left( \frac{D_{\ir ers}}{2 h}\right)^2 } }\right)} $ \\ 


$D \ll 2h$ $C_b' = \frac{2 \pi \epsilon_0}{\ln \left( \frac{D_{\ir ers}}{ r_B} \right) }$ \\ 


\subsubsection{Ohmscher Querleitwert $G'$}

spezifische Arbeitsverluste: $P_V' = 3 \cdot \left( \frac{U_n}{\sqrt{3}} \right)^2 \cdot G_b' = U_n^2 \cdot G_b'$ \\

Betriebsableitbelag: $G_b' = \frac{P_V'}{U_n^2}$

}
\sectionbox{
\subsection{Kabel}


Querschnitt: $Q = (r_a^2 - r_i^2) \pi$ \\ 




\subsubsection{Widerstandsbelag $R'$}

\subsubsection{Induktivitätsbelag $L'$}
Betriebsinduktivitäsbelag: $L_b' = \frac{L_B}{l} = \left( 2 \ln \frac{D}{r}+ \frac{1}{2} \right) \cdot 10^{-4} \si{\henry \per \kilo \meter}$ \\

Betriebsreaktanzbelag: $X_b ' = \omega	L_b ' $ \\ 

Induktivitätsbelag eines Hohlleiters:

$\omega L_{b_{\ir HL}}' = ( 0,96 + 0,051 \frac{r_a - r_i}{r_a})$ für $0 < \frac{r_a - r_i}{r_a} < 0,6$
\\

\subsubsection{Kapazitätsbelag $C'$}

Radialfeldkabel: $C_b' = \frac{2 \pi \epsilon}{\ln \left( \frac{R_a}{R_i} \right )}$ \\ 


\subsubsection{Ohmscher Querleitwert}
$G_b' = \tan \delta \cdot \omega C_b'$
}
\section{Leitung im stationären und nichtstationären Betrieb}
\sectionbox{
\subsection{Querkompensation}

 $\beta = \sqrt{ L' C' }$ \\ 
$\vartheta_{\ir nat} = \beta l $

$Z_w = \sqrt{\frac{\omega L'}{\omega C'}}$ \\ 

$\vec Z_l = Z_w \j \sin ( \beta l)$ \\

$\frac{Y_q}{2} = \frac{1}{Z_w} \j \tan \left( \frac{\beta l}{2} \right)$

$P_{\ir nat} = \frac{U_n^2}{Z_w}$

\boxed{
Q_2 = \frac{P_{\ir nat}}{\sin ( \beta l )} \left[ \sqrt{1 - \left( \frac{p_2}{P_{\ir nat}} \sin (\beta l) \right)^2} - \cos (\beta l) \right]}
}
\section{Transformatoren}
\end{multicols*}
\end{document}